\documentclass[12pt]{article}

\usepackage[margin=1in]{geometry}

% Math and Algos
\usepackage{amsmath, amsthm, amsfonts, amssymb}
\usepackage{mathtools}
\usepackage{algorithmicx, algpseudocode}
\usepackage{verbatim}
\newcommand\R{\mathbb{R}}
\newcommand\N{\mathbb{N}}
\newcommand\Q{\mathbb{Q}}
\newcommand\Z{\mathbb{Z}}
\newcommand{\sgn}{\operatorname{sgn}}
\newtheorem{theorem}{Theorem}[section]
\theoremstyle{remark}
\newtheorem*{remark}{Remark}

% Graphics
\usepackage{graphicx}
\usepackage{color}
\graphicspath{{./images/}}
\usepackage{booktabs}

% Hyperlinks
\usepackage{hyperref}
\hypersetup{
    linkcolor=cyan
}

% Aesthetics
\usepackage{enumitem}
\allowdisplaybreaks
\hfuzz=14pt

\begin{document}

\title{M 361K Homework 4}
\author{Ishan Shah}
\date{\today}
\maketitle

%%%%%%%%%% 6.3 %%%%%%%%%%
\section*{6.3}
\paragraph{13.} Try to use L'Hospital's Rule to find the limit of $\frac{\tan x}{\sec x}$ as $x \rightarrow(\pi / 2)^-$. Then evaluate directly by changing to sines and cosines.
\begin{proof}
    \begin{align*}
        \lim_{x \rightarrow (\pi / 2)^-} \frac{\tan x}{\sec x} &= \lim_{x \rightarrow (\pi / 2)^-} \frac{\sec^2x}{\sec x \tan x} \\
        &= \lim_{x \rightarrow (\pi / 2)^-} \frac{\sec x}{\tan x} \\
        &= \lim_{x \rightarrow (\pi / 2)^-} \frac{\sec x \tan x}{\sec^2x} \\
        &= \lim_{x \rightarrow (\pi / 2)^-} \frac{\tan x}{\sec x}
    \end{align*}
    After iteratively using L'Hospital's Rule, we find a cycle which means that we cannot use L'Hospital's Rule to find the limit. Instead, we can us sines and cosines.
    \begin{align*}
        \lim_{x \rightarrow (\pi / 2)^-} \frac{\tan x}{\sec x} &= \lim_{x \rightarrow (\pi / 2)^-} \frac{\frac{\sin x}{\cos x}}{\frac{1}{\cos x}} \\
        &= \lim_{x \rightarrow (\pi / 2)^-} \sin x \\
        &= 1
    \end{align*}
    Thus, $\lim_{x \rightarrow (\pi / 2)^-} \frac{\tan x}{\sec x} = 1$.
\end{proof}

\paragraph{14.} Show that if $c>0$, then $\lim_{x \rightarrow c} \frac{x^c-c^x}{x^x-c^c}=\frac{1-\ln c}{1+\ln c}$.
\begin{proof}
    \begin{align*}
        \lim_{x \to c} \frac{x^c-c^x}{x^x-c^c} &= \lim_{x \to c} \frac{cx^{c - 1} - c^x \ln c}{x^x(\ln x + 1) - 0} \\
        &= \frac{cc^{c - 1} - c^c \ln c}{c^c(\ln c + 1)} \\
        &= \frac{c^c(1 - \ln c)}{c^c(1 + \ln c)} \\
        &= \frac{1 - \ln c}{1 + \ln c}
    \end{align*}
    Thus, $\lim_{x \to c} \frac{x^c-c^x}{x^x-c^c} = \frac{1 - \ln c}{1 + \ln c}$.
\end{proof}

%%%%%%%%%% 6.4 %%%%%%%%%%
\section*{6.4}
\paragraph{11.} If $x \in[0,1]$ and $n \in \mathbb{N}$, show that
$$
\left|\ln (1+x)-\left(x-\frac{x^2}{2}+\frac{x^3}{3}+\cdots+(-1)^{n-1} \frac{x^n}{n}\right)\right|<\frac{x^{n+1}}{n+1} .
$$
Use this to approximate $\ln 1.5$ with an error less than $0.01$. Less than $0.001$.
\begin{proof}
    We can start by finding the first few terms of the Taylor series of $\ln (1 + x)$. We need the derivatives of $\ln (1 + x)$ up to $n$.
    \begin{align*}
        f(x) &= \ln (1 + x) \implies f(0) = 0 \\
        f'(x) &= \frac{1}{1 + x} \implies f'(0) = 1 \\
        f''(x) &= -\frac{1}{(1 + x)^2} \implies f''(0) = -1 \\
        \vdots \\
        f^{(n)}(x) &= (-1)^{n - 1} \frac{(n - 1)!}{(1 + x)^n} \implies f^{(n)}(0) = (-1)^{n - 1} (n - 1)!
    \end{align*}
    Now, we can find the Taylor series of $\ln (1 + x)$.
    \begin{align*}
        \ln (1 + x) &= \sum_{n = 0}^{\infty} \frac{(-1)^{n - 1} (n - 1)!}{n!} x^n \\
        &= x - \frac{x^2}{2} + \frac{x^3}{3} - \frac{x^4}{4} + \frac{x^5}{5} - \frac{x^6}{6} + \cdots
    \end{align*}
    Then, we have
    \begin{align*}
        \left|\ln(1 + x) - \left(x - \frac{x^2}{2} + \frac{x^3}{3} + \cdots + (-1)^{n - 1} \frac{x^n}{n}\right)\right| &= \left|x^{n + 1} \frac{f^{n + 1}(\varepsilon)}{(n + 1)!}\right| \\
        &= \left|(-1)^n x^{n + 1} \frac{n!}{(n + 1)!(1 + \varepsilon)^{n + 1}}\right| \\
        &= \left|\frac{x^{n + 1}}{(n + 1)(1 + \varepsilon)^{n + 1}}\right| \\
        &< \frac{x^{n + 1}}{n + 1}
    \end{align*}
    Now, we can use this to approximate $\ln 1.5$. We can let $x = 0.5$ and bound our error with $$\frac{x^{n + 1}}{n + 1}$$
    \begin{itemize}
        \item Error $< 0.01$: $n = 4$
        \begin{align*}
            \frac{0.5^{5}}{5} &= 0.006 < 0.01 \\
            \ln 1.5 &= 0.5 - \frac{0.5^2}{2} + \frac{0.5^3}{3} - \frac{0.5^4}{4} \approx 0.401
        \end{align*}
        \item Error $< 0.001$: $n = 7$
        \begin{align*}
            \frac{0.5^{8}}{8} &= 0.0005 < 0.001 \\
            \ln 1.5 &= 0.5 - \frac{0.5^2}{2} + \frac{0.5^3}{3} - \frac{0.5^4}{4} + \frac{0.5^5}{5} - \frac{0.5^6}{6} + \frac{0.5^7}{7} \approx 0.405
        \end{align*}
    \end{itemize}
\end{proof}

\paragraph{13.} Calculate $e$ correct to 7 decimal places.
\begin{proof}
    We can use the Taylor series of $e^x$ to approximate $e$.
    \begin{align*}
        e^x &= \sum_{n = 0}^{\infty} \frac{x^n}{n!} \\
        &= 1 + x + \frac{x^2}{2} + \frac{x^3}{3!} + \frac{x^4}{4!} + \frac{x^5}{5!} + \frac{x^6}{6!} + \frac{x^7}{7!} + \cdots
    \end{align*}
    We only need the first 11 terms of the Taylor series to get $e$ correct to 7 decimal places.
    \begin{align*}
        e &= 1 + 1 + \frac{1}{2} + \frac{1}{6} + \frac{1}{24} + \frac{1}{120} + \frac{1}{720} + \frac{1}{5040} + \frac{1}{40320} + \frac{1}{362880} + \frac{1}{3628800} \\
        &= 2.7182818
    \end{align*}
\end{proof}

%%%%%%%%%% 7.1 %%%%%%%%%%
\section*{7.1}
\paragraph{2.} If $f(x):=x^2$ for $x \in[0,4]$, calculate the following Riemann sums, where $\dot{\mathcal{P}}_i$ has the same partition points as in Exercise 1, and the tags are selected as indicated. $\mathcal{P}_2 := (0, 2, 3, 4)$.
\begin{itemize}
    \item $\dot{\mathcal{P}}_2$ with the tags at the left endpoints of the subintervals.
    \begin{align*} 
        S(f, \dot{\mathcal{P}}) &=\sum_{i=1}^n f\left(t_i\right)\left(x_i-x_{i-1}\right) \\
        &=f(0)\left(x_1-x_0\right)+f(2)\left(x_2-x_1\right)+f(3)\left(x_3-x_2\right) \\
        &=0^2(2-0)+2^2(3-2)+3^2(4-3) \\
        &=0 \cdot 2+4 \cdot 1+9 \cdot 1=13 
    \end{align*}
    \item $\dot{\mathcal{P}}_2$ with the tags at the right endpoints of the subintervals.
    \begin{align*}
        S(f, \dot{\mathcal{P}}) &=\sum_{i=1}^n f\left(t_i\right)\left(x_i-x_{i-1}\right) \\
        &=f(2)\left(x_1-x_0\right)+f(3)\left(x_2-x_1\right)+f(4)\left(x_3-x_2\right) \\
        &=2^2(2-0)+3^2(3-2)+4^2(4-3) \\
        &=4 \cdot 2+9 \cdot 1+16 \cdot 1=33
    \end{align*}
\end{itemize}

\paragraph{6b.} Let $h(x):=2$ if $0 \leq x<1, h(1):=3$ and $h(x):=1$ if $1<x \leq 2$. Show that $h \in \mathcal{R}[0,2]$ and evaluate its integral.
\begin{proof}

\end{proof}

\paragraph{8.} If $f \in \mathcal{R}[a, b]$ and $|f(x)| \leq M$ for all $x \in[a, b]$, show that $\left|\int_a^b f\right| \leq M(b-a)$.
\begin{proof}
    We have that $-M \leq f(x) \leq M$ for all $x \in[a, b]$. Therefore, we can write $$-M(b - a) = \int_a^b -M \leq \int_a^b f \leq \int_a^b M = M(b - a)$$ Thus, $\left|\int_a^b f\right| \leq M(b - a)$.
\end{proof}

\paragraph{10.} Let $g(x):=0$ if $x \in[0,1]$ is rational and $g(x):=1 / x$ if $x \in[0,1]$ is irrational. Explain why $g \notin \mathcal{R}[0,1]$. However, show that there exists a sequence $(\dot{\mathcal{P}}_n)$ of tagged partitions of $[a, b]$ such that $||\dot{\mathcal{P}}_n|| \rightarrow 0$ and $\lim_n S(g ; \dot{\mathcal{P}}_n)$ exists.
\begin{proof}
    Because $g(x)$ is not bounded on $[0, 1]$, $g \notin \mathcal{R}[0, 1]$. Now, let $$\dot{\mathcal{P}_n} = \{([\frac{i - 1}{n}, \frac{i}{n}], \frac{i}{n})\}_{i = 1}^n \forall n \in \mathbb{N}$$ Then, $||\dot{\mathcal{P}}_n|| = \frac{1}{n}$ so $\lim_{n \to \infty} ||\dot{\mathcal{P}}_n|| = \infty$. Now, $S(g, \dot{\mathcal{P}}_n) = \sum_{i = 1}^n \left(\frac{i}{n}\right) \left(\frac{i}{n} - \frac{i - 1}{n}\right) = 0$ since $g(x) = 0$ for all rational $x \in[0, 1]$. Thus, $\lim_n S(g, \dot{\mathcal{P}}_n) = 0$.
\end{proof}

\end{document}