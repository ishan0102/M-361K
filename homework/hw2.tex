\documentclass[12pt]{article}

\usepackage[margin=1in]{geometry}

% Math and Algos
\usepackage{amsmath, amsthm, amsfonts, amssymb}
\usepackage{mathtools}
\usepackage{algorithmicx, algpseudocode}
\usepackage{verbatim}
\newcommand\R{\mathbb{R}}
\newcommand\N{\mathbb{N}}
\newcommand\Q{\mathbb{Q}}
\newcommand\Z{\mathbb{Z}}
\newcommand{\sgn}{\operatorname{sgn}}
\newtheorem{theorem}{Theorem}[section]
\theoremstyle{remark}
\newtheorem*{remark}{Remark}

% Graphics
\usepackage{graphicx}
\usepackage{color}
\graphicspath{{./images/}}
\usepackage{booktabs}

% Hyperlinks
\usepackage{hyperref}
\hypersetup{
    linkcolor=cyan
}

% Aesthetics
\usepackage{enumitem}
\allowdisplaybreaks
\hfuzz=14pt

\begin{document}

\title{M 361K Homework 2}
\author{Ishan Shah}
\date{\today}
\maketitle

\section*{3.3}
\paragraph{5.} Let $y_1 := \sqrt{p}$, where $p > 0$, and $y_{n + 1} := \sqrt{p + y_n} \;\forall\; n \in \N$. Show that $(y_n)$ converges and find the limit.
\begin{proof}
    We want to show that $(y_n)$ is both monotonically increasing and bounded. First, we show that $(y_n)$ is monotonically increasing. We have that $y_1 = \sqrt{p}$ and $y_2 = \sqrt{p + y_1} = \sqrt{p + \sqrt{p}}$. Thus, $y_2^2 - y_1^2 = p + \sqrt{p} - p = \sqrt{p} > 0$. Thus, $y_2 > y_1$. We can continue this argument to show that $(y_{n + 1}) > (y_n)$ as $(y_{n + 1}) - (y_n) = (y_n) + (y_{n - 1}) > 0$. Thus, $(y_n)$ is monotonically increasing.
    
    Now, we show that $(y_n)$ is bounded. We have that $y_1 = \sqrt{p} < 1 + 2\sqrt{p}$. Next, we have that $(y_n) < 1 + 2\sqrt{p}$. From here, $(y_{n + 1}^2) < p + (y_n) < p + 1 + \sqrt{p} < (\sqrt{p} + 1)^2 < 1 + 2\sqrt{p}$. Thus, since $(y_n) < 1 + 2\sqrt{p}$, $(y_n)$ is bounded, so $(y_n)$ converges to some value $c$.

    Now, we want to find this value $c$. We have that $\lim_{n \to \infty} (y_n) = \lim_{n \to \infty} \sqrt{p + y_{n - 1}} = \sqrt{p + \lim_{n \to \infty} (y_{n - 1})}$. Then, $c = \sqrt{p + c}$. Thus, $c^2 = p + c \implies c = \frac{1}{2}(1 + \sqrt{1 + 4p})$.
\end{proof}

\paragraph{8.} Let $(a_n)$ be an increasing sequence, $(b_n)$ be a decreasing sequence, and assume that $a_n \leq b_n \;\forall\; n \in \N$. Show that $\lim(a_n) \leq \lim(b_n)$.
\begin{proof}
    Since $(b_n)$ is decreasing, $b_1$ is the upper bound of $(b_n)$ and also the upper bound of $(a_n)$ since $a_n \leq b_n \;\forall\; n \in \N$. Thus, $(a_n)$ is bounded below by $a_1$ and above by $b_1$, and $(a_n)$ is monotonic, so it must converge to some limit. Similarly, $(b_n)$ is bounded below by $a_1$ and above by $b_1$, and $(b_n)$ is monotonic, so it must converge to some limit.

    Now, we can use Theorem 3.2.5 which states that for two convergent sequences $(a_n)$ and $(b_n)$, if $a_n \leq b_n \;\forall\; n \in \N$, then $\lim(a_n) \leq \lim(b_n)$. Thus, $\lim(a_n) \leq \lim(b_n)$.
\end{proof}

\paragraph{12.} Establish the convergence and find the limits of the following sequences.
\begin{enumerate}[label=(\alph*)]
    \item $((1 + 1/n)^{n + 1})$
    \begin{align*}
        \lim_{n \to \infty} (1 + 1/n)^{n + 1} &= \lim_{n \to \infty} (1 + 1/n)^n \cdot (1 + 1/n) \\
        &= \lim_{n \to \infty} (1 + 1/n)^n \cdot \lim_{n \to \infty} (1 + 1/n) \\
        &= e \cdot 1 \\
        &= e
    \end{align*}

    \item $((1 + 1/n)^{2n})$
    \begin{align*}
        \lim_{n \to \infty} (1 + 1/n)^{n + 1} &= \lim_{n \to \infty} (1 + 1/n)^n \cdot (1 + 1/n)^n \\
        &= \lim_{n \to \infty} (1 + 1/n)^n \cdot \lim_{n \to \infty} (1 + 1/n)^n \\
        &= e \cdot e \\
        &= e^2
    \end{align*}
\end{enumerate}

\section*{3.4}
\paragraph{1.} Give an example of an unbounded sequence that has a convergent subsequence.
\begin{proof}
    Let there be some sequence $(x_n)$ where $(x_n) = 1$ if $n$ is even and $(x_n) = n$ if $n$ is odd. Then, $(x_n)$ is an unbounded sequence, yet the susbequence $(x_{2n})$ is convergent as it is bounded and monotonic. Thus, $(x_n)$ has a convergent subsequence.
\end{proof}

\paragraph{4b.} Show that the sequence $(\sin n\pi/4)$ is divergent.
\begin{proof}
    Let $(x_n) = \sin n\pi/4$. We want to show that $(x_n)$ has two convergent subsequences whose limits are not equal. Let $(y_n) = (x_{4n})$ and $(z_n) = (x_{8n + 1})$ be subsequences of $(x_n)$.

    Then, $(y_n) = \sin (4n\pi/4) = \sin (n\pi) = 0$ and $(z_n) = \sin((8n + 1)\pi/4) = \sin(2n\pi + \pi/4) = \sin(\pi/4) = \sqrt{2}/2$. Thus, $(y_n)$ and $(z_n)$ are both convergent subsequences of $(x_n)$, yet their limits are not equal. Therefore, $(x_n)$ is divergent.
\end{proof}

\paragraph{10.} Let $(x_n)$ be a bounded subsequence and for each $n \in \N$, let $s_n := \sup\{x_k : k \geq n\}$ and $S := \inf\{s_n\}$. Show that there exists a subsequence of $(x_n)$ that converges to $S$.
\begin{proof}
    For $\epsilon > 0$, there exists some $n \in \N$ such that $s_n < S + \epsilon$. We can choose $\epsilon = 1$ and $m_1$ such that $s_{m_1} - 1 < S + 1$ and $k_1 \geq m_1$ such that $s_{m_1} - 1 < x_{k_1} < s_{m_1}$ since $s_{m_1} = \sup\{x_n : k \geq m_1\}$.

    Then, we can choose some $m_n > m_{n - 1}$ such that $S \leq s_{m_n} < S + \frac{1}{n}$ and $k_n \geq m_n$ and $k_n > k_{n - 1}$ such that $s_{m_n} - \frac{1}{n} < x_{k_n} < s_{m_n}$. Now, we have a subsequence $(x_{k_n})$ of $(x_n)$ where $|x_{k_n} - S| \leq \frac{1}{n}$. Finally, we know that $\lim_{n \to \infty} \frac{1}{n} = 0$. Thus, $\lim_{n \to \infty} x_{k_n} = S$.
\end{proof}

\paragraph{12.} Show that if $(x_n)$ is unbounded, then there exists a subsequence $(x_{n_k})$ such that $\lim_{k \to \infty} (1/x_{n_k}) = 0$.
\begin{proof}
    Let $(x_n)$ be unbounded. Then, there exists some $n_1 \in \N$ such that $|x_{n_1}| \geq 1$. There also exists some $n_2 > n_1 \in \N$ such that $|x_{n_2}| \geq 2$. We can continue this with some arbitrary sequence $n_i \in \N$ such that $|x_{n_k}| \geq k$ for all $k \in \N$ because this sequence is unbounded. Then, $$0 \leq \frac{1}{|x_{n_k}|} \leq \frac{1}{k}$$ We know that $\lim_{k \to \infty} \frac{1}{k} = 0$, so $\lim_{k \to \infty} \frac{1}{x_{n_k}} = 0$ by Theorem 3.2.7 (Squeeze Theorem).
\end{proof}

\section*{3.5}
\paragraph{2.} Show directly from the definition that the following are Cauchy sequences.
\begin{enumerate}[label=(\alph*)]
    \item $(\frac{n + 1}{n})$ \\\null\\
    We want to show that $\exists N \in \N$ such that $m, n > N$ and $|S_m - S_n| < \epsilon$. Let there be some arbitrary $N$ such that $\frac{1}{N} = \frac{\epsilon}{2}$ and some $m > n \geq N$. Then,
    \begin{align*}
        \left\lvert\frac{m + 1}{m} - \frac{n + 1}{n}\right\rvert &= \left\lvert\frac{1}{m} - \frac{1}{n}\right\rvert \\
        &\leq \frac{1}{m} + \frac{1}{n} \\
        &\leq \frac{2}{n} \\
        &\leq \frac{2}{N} \\
        &< \epsilon
    \end{align*}
    Thus, $(\frac{n + 1}{n})$ is a Cauchy sequence.

    \item $(1 + \frac{1}{2!} + \cdots + \frac{1}{n!})$
    We want to show that $\exists N \in \N$ such that $m, n > N$ and $|S_m - S_n| < \epsilon$. Let there be some arbitrary $N$ such that $\frac{1}{2^N} < \epsilon$ and some $m > n \geq N$. Then,
    \begin{align*}
        \left\lvert\left(1 + \frac{1}{2!} + \cdots + \frac{1}{m!}\right) - \left(1 + \frac{1}{2!} + \cdots + \frac{1}{n!}\right)\right\rvert &= \frac{1}{(n + 1)!} + \cdots + \frac{1}{m!} \\
        &\leq \frac{1}{2^{n + 1}} + \cdots + \frac{1}{2^m} \\
        &= \frac{1}{2^n} \left(\frac{1}{2} + \frac{1}{2^2} + \cdots + \frac{1}{2^{m - n}}\right) \\
        &\leq \frac{1}{2^n} \left(\frac{1}{2} + \frac{1}{4} + \cdots\right) \\
        &= \frac{1}{2^n} * 1 \\
        &\leq \frac{1}{2^N} \\
        &< \epsilon
    \end{align*}
    Thus, $(1 + \frac{1}{2!} + \cdots + \frac{1}{n!})$ is a Cauchy sequence.
\end{enumerate}

\paragraph{7.} Let $(x_n)$ be a Cauchy sequence such that $x_n$ is an integer for every $n \in \N$. Show that $(x_n)$ is ultimately constant.
\begin{proof}
    Let $\epsilon = 1$. Then, there must exist some $N$ such that $m, n > N$ and $|x_m - x_n| < \epsilon$. However, since $x_m$ and $x_n$ are integers, $x_m = x_n$ in order to satisfy the Cauchy condition. Thus, $(x_n)$ is ultimately constant.
\end{proof}

\paragraph{8.} Show directly that a bounded, monotone increasing sequence is a Cauchy sequence.
\begin{proof}
    Let there be some bounded, monotone increasing sequence $(x_n)$ that has some supremum $M$. We know that there exists some $n_0$ and $\epsilon > 0$ such that $M - \epsilon < x_{n_0} < M$. Since $(x_n)$ is increasing, we also know that there exists some $n_1$ and $n_2$ such that $x_{n_0} \leq x_{n_1} \leq x_{n_2}$. We can combine these inequalities to get that $M - \epsilon < x_{n_0} \leq x_{n_1} \leq x_{n_2} < M$.

    Now, we want to show that $|x_{n_1} - x_{n_2}| < \epsilon$ to satisfy the definition of a Cauchy sequence. Since we know that $M - \epsilon < x_{n_1} < M$ and $M - \epsilon < x_{n_2} < M$, we know that $-M < -x_{n_2} < \epsilon - M$. Then, we can combine our equations like so:
    \begin{align*}
        M - \epsilon - M &< x_{n_1} - x_{n_2} < M + \epsilon - M \\
        &\implies \epsilon < x_{n_1} - x_{n_2} < \epsilon \\
        &\implies |x_{n_1} - x_{n_2}| < \epsilon
    \end{align*}
    Thus, $(x_n)$ is a Cauchy sequence.
\end{proof}

\section*{4.1}
\paragraph{2.} Determine a condition on $|x - 4|$ to assure the following inequalities. \\\null\\
We can break down our original equation to yield a more useful form assuming that $x \geq 0$:
\begin{align*}
    x - 4 &= (\sqrt{x} + 2)(\sqrt{x} - 2) \\
    |\sqrt{x} - 2| &= \frac{|x - 4|}{\sqrt{x} + 2} \\
    |\sqrt{x} - 2| &\leq \frac{|x - 4|}{2}
\end{align*}
We can use this new form to easily solve these inequalities.
\begin{enumerate}[label=(\alph*)]
    \item $|\sqrt{x} - 2| < \frac{1}{2}$
    \begin{proof}
        Let $|x - 4| < 1$. Then, $|\sqrt{x} - 2| < \frac{1}{2}$.
    \end{proof}

    \item $|\sqrt{x} - 2| < 10^{-2}$
    \begin{proof}
        Let $|x - 4| < 2 \cdot 10^{-2}$. Then, $|\sqrt{x} - 2| < 10^{-2}$.
    \end{proof}
    
\end{enumerate}

\paragraph{6.} Let $I$ be an interval in $\R$, let $f: I \to \R$, and let $c \in I$. Suppose that there exists constants $K$ and $L$ such that $|f(x) - L| \leq K|x - c|$ for $x \in I$. Show that $\lim_{x \to c} f(x) = L$.
\begin{proof}
    We want to show that $\epsilon > 0$ as we can find some $\delta > 0$ such that if $|x - c| < \delta$, then $|f(x) - L| < \epsilon$. Because $|f(x) - L| \leq K|x - c|$, we have $\delta = \frac{\epsilon}{K}$. If $|x - c| < \frac{\epsilon}{K}$, then $|f(x) - L| \leq K|x - c| < \epsilon$. Thus, $|f(x) - L| < \epsilon$, so $\lim_{x \to c} f(x) = L$.
\end{proof}

\paragraph{9a.} Use either the $\epsilon-\delta$ definition of a limit or the Sequential Criterion for limits to establish that $\lim_{x \to 2} \frac{1}{1 - x} = -1$.
\begin{proof}
    We assume we have some $(s_n) \to 2$. Then, $$\lim_{n \to \infty} \frac{1}{1 - x} = \frac{1}{1 - 2} = -1$$ Then, by the Sequential Criterion for limits, we have that $\lim_{x \to 2} \frac{1}{1 - x} = -1$.
\end{proof}

\paragraph{10a.} Use the definition of the limit to show that $\lim_{x \to 2} (x^2 + 4x) = 12$.
\begin{proof}
    Let $\delta = \min\{1, \frac{epsilon}{9}\}$ and $\epsilon > 0$. We also have $x$ such that $|x - 2| < \delta$. We want to show that the difference at any two arbitrary points is less than $\epsilon$. Then,
    \begin{align*}
        |x^2 + 4x - 12| &= |(x - 2)(x + 6)| \\
        &\leq |x - 2||x + 6| \\
        &\leq \delta|x + 6| \\
        &= \delta|x - 2 + 8| \\
        &\leq \delta|\delta + 8| \\
        &\leq \delta(1 + 8) \\
        &= \delta(9) \\
        &\leq \epsilon
    \end{align*}
    Thus, $\lim_{x \to 2} (x^2 + 4x) = 12$.
\end{proof}

\paragraph{15.} Let $f: \R \to \R$ be defined by setting $f(x) := x$ if $x$ is rational, and $f(x) = 0$ if $x$ is irrational.
\begin{enumerate}[label=(\alph*)]
    \item Show that $f$ has a limit at $x = 0$.
    \begin{proof}
        For some $\epsilon > 0$, choose $\delta = \epsilon$ such that $|x - 0| = |x| < \delta$. Then, $|f(x) - 0| = |f(x)| \leq |x| \leq \delta = \epsilon$. Thus, $f(x)$ has a limit at $x = 0$.
    \end{proof}

    \item Use a sequential argument to show that if $c \neq 0$, then $f$ does not have a limit at $c$.
    \begin{proof}
        Let there be some $(x_n) \in \Q$ and $(y_n) \in \R \setminus \Q$ such that they both converge to $c$ where $c \neq 0$. We know that $f(x_n) = x_n$ and $f(y_n) = 0$ from the definition of the function. Thus, we have two convergent subsequences that do not converge to the same limit. Therefore, $f$ does not have a limit at $c$.
    \end{proof}
\end{enumerate}

\section*{4.2}
\paragraph{1.} Apply Theorem 4.2.4 to determine the following limits:
\begin{enumerate}[label=(\alph*)]
    \item $\lim_{x \to 1} (x + 1)(2x + 3)$ where $(x \in \R)$
    \begin{proof}
        \begin{align*}
            \lim_{x \to 1} (x + 1)(2x + 3) &= \lim_{x \to 1} (x + 1) \cdot \lim_{x \to 1} (2x + 3) \\
            &= 2 \cdot 5 \\
            &= 10
        \end{align*}
        Thus, $\lim_{x \to 1} (x + 1)(2x + 3) = 10$.
    \end{proof}

    \item $\lim_{x \to 1} \frac{x^2 + 2}{x^2 - 2}$ where $(x > 0)$
    \begin{proof}
        \begin{align*}
            \lim_{x \to 1} \frac{x^2 + 2}{x^2 - 2} &= \frac{\lim_{x \to 1} (x^2 + 2)}{\lim_{x \to 1} (x^2 - 2)} \\
            &= \frac{3}{-1} \\
            &= -3
        \end{align*}
        Thus, $\lim_{x \to 1} \frac{x^2 + 2}{x^2 - 2} = -3$.
    \end{proof}

    \item $\lim_{x \to 2} (\frac{1}{x + 1} - \frac{1}{2x})$ where $(x > 0)$
    \begin{proof}
        \begin{align*}
            \lim_{x \to 2} \left(\frac{1}{x + 1} - \frac{1}{2x}\right) &= \lim_{x \to 2} \left(\frac{1}{x + 1}\right) - \lim_{x \to 2} \left(\frac{1}{2x}\right) \\
            &= \frac{1}{3} - \frac{1}{4} \\
            &= \frac{1}{12}
        \end{align*}
        Thus, $\lim_{x \to 2} (\frac{1}{x + 1} - \frac{1}{2x}) = \frac{1}{12}$.
    \end{proof}

    \item $\lim_{x \to 0} \frac{x + 1}{x^2 + 2}$ where $(x \in \R)$
    \begin{proof}
        \begin{align*}
            \lim_{x \to 0} \frac{x + 1}{x^2 + 2} &= \frac{\lim_{x \to 0} (x + 1)}{\lim_{x \to 0} (x^2 + 2)} \\
            &= \frac{1}{2}
        \end{align*}
        Thus, $\lim_{x \to 0} \frac{x + 1}{x^2 + 2} = \frac{1}{2}$.
    \end{proof}
\end{enumerate}

\paragraph{4.} Prove that $\lim_{x \to 0} \cos(1/x)$ does not exist but that $\lim_{x \to 0} x \cos(1/x) = 0$.
\begin{proof}
    First, we will show that $\lim_{x \to 0} \cos(1/x)$ does not exist. Let $(x_n) = \frac{1}{n + \pi/2}$ and $(y_n) = \frac{1}{n + 2\pi}$. Then, $\lim_{n \to \infty} (x_n) = 0$ and $\lim_{n \to \infty} (y_n) = 0$. $\forall n \in \N$, $\cos(1/x_n) = \cos(n + \pi/2) = 0$ and $\cos(1/y_n) = \cos(n + 2\pi) = 1$. Thus, we have two convergent subsequences that do not converge to the same limit, so $\lim_{x \to 0} \cos(1/x)$ does not exist.

    Now, we will show that $\lim_{x \to 0} x \cos(1/x) = 0$. Let $\epsilon > 0$. We know $\exists \delta = \epsilon$ such that $|x - 0| = |x| < \delta$. Then,
    \begin{align*}
        |x \cos(1/x) - 0| &= |x \cos(1/x)| \\
        &\leq |x| \\
        &\leq \delta \\
        &= \epsilon
    \end{align*}
    Thus, $\lim_{x \to 0} x \cos(1/x) = 0$.
\end{proof}

\paragraph{6.} Use the definition of the limit to prove that if $\lim_{x \to c} f = L$ and $\lim_{x \to c} g = M$, then $\lim_{x \to c} (f + g) = L + M$.
\begin{proof}
    Let $\epsilon > 0$. Then, we know there exists some $\delta_f, \delta_g > 0$ such that $|x - c| < \delta_f$ and $|x - c| < \delta_g \implies |f(x) - L| < \frac{\epsilon}{2}$ and $|g(x) - M| < \frac{\epsilon}{2}$.
    
    We choose $\delta = \max\{\delta_f, \delta+g\}$. Then, for $|x - c| < \delta$, we have
    \begin{align*}
        |f(x) + g(x) - (L + M)| &= |(f(x) - L) + (g(x) - M)| \\
        &\leq |f(x) - L| + |g(x) - M| \\
        &< \frac{\epsilon}{2} + \frac{\epsilon}{2} \\
        &= \epsilon
    \end{align*}
    Thus, $\lim_{x \to c} (f + g) = L + M$.
\end{proof}

\paragraph{10.} Give examples of functions $f$ and $g$ such that $f$ and $g$ do not have limits at a point $c$, but such that both $f + g$ and $fg$ have limits at $c$.
\begin{proof}
    Let $f(x) = \sgn(x)$ and $g(x) = -\sgn(x)$ and $c = 0$. Then, $f$ and $g$ do not have limits at $c$, but $f + g = 0$ and $fg = -1$. Thus, $f + g$ and $fg$ have limits at $c$.
\end{proof}
\pagebreak

\paragraph{11.} Determine whether the following limits exist at $\R$.
\begin{enumerate}[label=(\alph*)]
    \item $\lim_{x \to 0} \sin(1/x^2)$ where $(x \neq 0)$.
    \begin{proof}
        Let $(x_n) = \frac{1}{\sqrt{n\pi}}$ and $(y_n) = \frac{1}{\sqrt{2\pi n + \pi/2}}$. Then, $\lim_{n \to \infty} (x_n) = 0$ and $\lim_{n \to \infty} (y_n) = 0$. $\forall n \in \N$, $\sin(1/x_n^2) = \sin(n\pi) = 0$ and $\sin(1/y_n^2) = \sin(2\pi n + \pi/2) = 1$. Thus, we have two convergent subsequences that do not converge to the same limit, so $\lim_{x \to 0} \sin(1/x^2)$ does not exist.
    \end{proof}

    \item $\lim_{x \to 0} x \sin(1/x^2)$ where $(x \neq 0)$.
    \begin{proof}
        We have some $\epsilon > 0$ and $\delta = \epsilon$ such that $|x - 0| = \delta$. Then,
        \begin{align*}
            |x \sin(1/x^2) - 0| &= |x \sin(1/x^2)| \\
            &\leq |x| \\
            &\leq \delta \\
            &= \epsilon
        \end{align*}
        Thus, $\lim_{x \to 0} x \sin(1/x^2) = 0$.
    \end{proof}

    \item $\lim_{x \to 0} \sgn \sin(1/x)$ where $(x \neq 0)$.
    \begin{proof}
        Let $(x_n) = \frac{1}{n\pi}$ and $(y_n) = \frac{1}{2\pi n + \pi/2}$. Then, $\lim_{n \to \infty} (x_n) = 0$ and $\lim_{n \to \infty} (y_n) = 0$. $\forall n \in \N$, $\sgn \sin(1/x_n) = \sgn \sin(n\pi) = 0$ and $\sgn \sin(1/y_n) = \sgn \sin(2\pi n + \pi/2) = 1$. Thus, we have two convergent subsequences that do not converge to the same limit, so $\lim_{x \to 0} \sgn \sin(1/x)$ does not exist.
    \end{proof}

    \item $\lim_{x \to 0} \sqrt{x} \sin(1/x^2)$ where $(x > 0)$.
    \begin{proof}
        We have some $\epsilon > 0$ and $\delta = \epsilon$ such that $|\sqrt{x} - 0| = \delta$. Then,
        \begin{align*}
            |\sqrt{x} \sin(1/x^2) - 0| &= |\sqrt{x} \sin(1/x^2)| \\
            &\leq |\sqrt{x}| \\
            &\leq \delta \\
            &= \epsilon
        \end{align*}
        Thus, $\lim_{x \to 0} \sqrt{x} \sin(1/x^2) = 0$.
    \end{proof}
\end{enumerate}

\end{document}