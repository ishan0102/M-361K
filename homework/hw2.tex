\documentclass[12pt]{article}

\usepackage[margin=1in]{geometry}

% Math and Algos
\usepackage{amsmath, amsthm, amsfonts, amssymb}
\usepackage{mathtools}
\usepackage{algorithmicx, algpseudocode}
\usepackage{verbatim}
\newcommand\R{\mathbb{R}}
\newcommand\N{\mathbb{N}}
\newcommand\Q{\mathbb{Q}}
\newcommand\Z{\mathbb{Z}}
\newcommand{\sgn}{\operatorname{sgn}}
\newtheorem{theorem}{Theorem}[section]
\theoremstyle{remark}
\newtheorem*{remark}{Remark}

% Graphics
\usepackage{graphicx}
\usepackage{color}
\graphicspath{{./images/}}
\usepackage{booktabs}

% Hyperlinks
\usepackage{hyperref}
\hypersetup{
    linkcolor=cyan
}

% Aesthetics
\usepackage{enumitem}
\allowdisplaybreaks
\hfuzz=14pt

\begin{document}

\title{M 361K Homework 2}
\author{Ishan Shah}
\date{\today}
\maketitle

\section*{3.3}
\paragraph{5.} Let $y_1 := \sqrt{p}$, where $p > 0$, and $y_{n + 1} := \sqrt{p + y_n} \;\forall\; n \in \N$. Show that $(y_n)$ converges and find the limit.

\paragraph{8.} Let $(a_n)$ be an increasing sequence, $(b_n)$ be a decreasing sequence, and assume that $a_n \leq b_n \;\forall\; n \in \N$. Show that $\lim(a_n) \leq \lim(b_n)$, and thereby deduce the Nested Intervals Property 2.5.2 from the Monotone Convergence Theorem 3.3.2.

\paragraph{12.} Establish the convergence and find the limits of the following sequences.
\begin{enumerate}[label=(\alph*)]
    \item $((1 + 1/n)^{n + 1})$
    \item $((1 + 1/n)^2n)$
\end{enumerate}

\section*{3.4}
\paragraph{1.} Give an example of an unbounded sequence that has a convergent subsequence.

\paragraph{4b.}
Show that the sequence $(\sin n\pi/4)$ is divergent.

\paragraph{10.}
Let $(x_n)$ be a bounded subsequence and for each $n \in \N$, let $s_n := \sup\{x_k : k \geq n\}$ and $S := \inf\{s_n\}$. Show that there exists a subsequence of $(x_n)$ that converges to $S$.

\paragraph{12.}
Show that if $(x_n)$ is unbounded, then there exists a subsequence $(x_{n_k})$ such that $\lim(1/x_{n_k}) = 0$.

\section*{3.5}
\paragraph{2.}
Show directly from the definition that the following are Cauchy sequences.
\begin{enumerate}[label=(\alph*)]
    \item $(\frac{n + 1}{n})$
    \item $(1 + \frac{1}{2!} + \cdots + \frac{1}{n!})$
\end{enumerate}

\paragraph{7.}
Let $(x_n)$ be a Cauchy sequence such that $x_n$ is an inteeger for every $n \in \N$. Show that $(x_n)$ is ultimately constant.

\paragraph{8.}
Show directly that a bounded, monotone increasing sequence is a Cauchy sequence.

\section*{4.1}
\paragraph{2.} Determine a condition on $|x - 4|$ that will assure that:
\begin{enumerate}[label=(\alph*)]
    \item $|\sqrt{x} - 2| < \frac{1}{2}$
    \item $|\sqrt{x} - 2| < 10^{-2}$
\end{enumerate}

\paragraph{6.} Let $I$ be an interval in $\R$, let $f: I \to \R$, and let $c \in I$. Suppose that there exists constants $K$ and $L$ such that $|f(x) - L| \leq K|x - c|$ for $x \in I$. Show that $\lim_{x \to c} f(x) = L$.

\paragraph{9a.} Use either the $\epsilon$-$\delta$ definition of a limit or the Sequential Criterion for limits to establish that $\lim_{x \to 2} \frac{1}{1 - x} = -1$.

\paragraph{10a.} Use the definition of the limit to show that $\lim_{x \to 2} (x^2 + 4x) = 12$.

\paragraph{15.} Let $f: \R \to \R$ be defined by setting $f(x) := x$ if $x$ is rational, and $f(x) = 0$ if $x$ is irrational.
\begin{enumerate}[label=(\alph*)]
    \item Show that $f$ has a limit at $x = 0$.
    \item Use a sequential argument to show that if $c \neq 0$, then $f$ does not have a limit at $c$.
\end{enumerate}

\section*{4.2}
\paragraph{1.} Apply theorem 4.2.4 to determine the following limits:
\begin{enumerate}[label=(\alph*)]
    \item $\lim_{x \to 1} (x + 1)(2x + 3)$ where $(x \in \R)$
    \item $\lim_{x \to 1} \frac{x^2 + 2}{x^2 - 2}$ where $(x > 0)$
    \item $\lim_{x \to 2} (\frac{1}{x + 1} - \frac{1}{2x})$ where $(x > 0)$
    \item $\lim_{x \to 0} \frac{x + 1}{x^2 + 2}$ where $(x \in \R)$
\end{enumerate}

\paragraph{4.} Prove that $\lim_{x \to 0} \cos(1/x)$ does not exist but that $\lim_{x \to 0} x \cos(1/x) = 0$.

\paragraph{6.} Use the definition of the limit to prove the first assertion in Theorem 4.2.4(a).

\paragraph{10.} Give examples of functions $f$ and $g$ such that $f$ and $g$ do not have limits at a point $c$, but such that both $f + g$ and $fg$ have limits at $c$.

\paragraph{11.} Determine whether the following limits exist at $\R$.
\begin{enumerate}[label=(\alph*)]
    \item $\lim_{x \to 0} \sin(1/x^2)$ where $(x \neq 0)$.
    \item $\lim_{x \to 0} x \sin(1/x^2)$ where $(x \neq 0)$.
    \item $\lim_{x \to 0} \sgn \sin(1/x)$ where $(x \neq 0)$.
    \item $\lim_{x \to 0} \sqrt{x} \sin(1/x^2)$ where $(x > 0)$.
\end{enumerate}

\end{document}