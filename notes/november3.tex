\section*{November 3}

\begin{theorem}
    Let $f$ be bounded on $[a, b]$ if $\mathcal{P}$ and $\mathcal{Q}$ are partitions of $[a, b]$ such that $\mathcal{Q}$ is a refinement of $\mathcal{P}$. Then, $$L(f, \mathcal{P}) \leq L(f, \mathcal{Q}) \leq U(f, \mathcal{Q}) \leq U(f, \mathcal{P})$$
\end{theorem}
\begin{proof}
    We know that $\mathcal{Q}$ will contain more points than $\mathcal{P}$. $\mathcal{P}$ is described by $m_k \cdot (x_k - x_{k - 1})$ while $\mathcal{Q}$ is described by $m_x \cdot (x^\star - x_{k - 1}) + m_x \cdot (x_k - x^\star)$.
\end{proof}

\begin{theorem}
    Let $\mathcal{P}$ and $\mathcal{Q}$ be partitions of $[a, b]$. Then $$L(f, \mathcal{P}) \leq U(f, \mathcal{Q})$$
\end{theorem}
\begin{proof}
    Let $\mathcal{P}$ and $\mathcal{Q}$ be partitions of $f$. Then $\mathcal{P} \cup \mathcal{Q}$ is a refinement of $\mathcal{P}$ and $\mathcal{Q}$. Thus, $$L(f, \mathcal{P}) \leq (f, \mathcal{P} \cup \mathcal{Q}) \leq U(f, \mathcal{P} \cup \mathcal{Q}) \leq U(f, \mathcal{Q})$$
\end{proof}

\begin{theorem}
    Let $f$ be bounded on $[a, b]$. Then, $L(f) \leq U(f)$.
\end{theorem}
\begin{proof}
    Let $\mathcal{P}$ and $\mathcal{Q}$ be partitions of $[a, b]$. Then by the previous theorem, $U(f, \mathcal{Q})$ is an upper bound for $$S = \{L(f, \mathcal{P}) : \text{$\mathcal{P}$ is a partition of $[a, b]$}\}$$
    So, $U(f, \mathcal{Q})$ is at least as large as $\sup S = L(f)$. That is, $L(f) \leq U(f, \mathcal{Q})$ for each partition $\mathcal{Q}$. Then, $$L(f) \leq \inf\{U(f, \mathcal{Q}) : \text{$\mathcal{Q}$ is a partition of $[a, b]$}\} = U(f)$$
    Therefore, $L(f) \leq U(f)$.
\end{proof}

\begin{example}
    $f(x) = x^2$ on $[0, 1]$ with partition $\mathcal{\mathcal{P}}_n = \{0, \frac{1}{n}, \frac{2}{n}, \ldots, \frac{n - 1}{n}, 1\}$.

    \begin{align*}
        M_i &= \sup \left\{f(x) : x \in \left[\frac{i - 1}{n}, \frac{i}{n}\right]\right\} = \left(\frac{i^2}{n^2}\right) \\
        M_i &= \inf \left\{f(x) : x \in \left[\frac{i - 1}{n}, \frac{i}{n}\right]\right\} = \left(\frac{i - 1}{n}\right)^2 \\
        U(f, \mathcal{\mathcal{P}}_n) &= \sum_{i = 1}^n M_i \cdot \Delta x_i = \sum_{i = 1}^n \left(\frac{i}{n}\right)^2 \cdot \frac{1}{n} = \frac{1}{n^3} \sum_{i = 1}^n i^2 = \left[\frac{1}{n^3} \cdot \frac{n(n + 1)(2n + 1)}{6}\right] \\
        L(f, \mathcal{\mathcal{P}}_n) &= \sum_{i = 1}^n m_i \cdot \Delta x_i = \sum_{i = 1}^n \left(\frac{i - 1}{n}\right)^2 \cdot \frac{1}{n} = \frac{1}{n^3} \sum_{i = 1}^n (i - 1)^2 = \left[\frac{1}{n^3} \cdot \frac{n(n - 1)(2n - 1)}{6}\right]
    \end{align*}
    Then, $\lim_{n \to \infty} U(f, \mathcal{\mathcal{P}}_n) = \frac{1}{3}$ and $\lim_{n \to \infty} L(f, \mathcal{\mathcal{P}}_n) = \frac{1}{3}$. Thus, $U(f) \leq \frac{1}{3}$ and $L(f) \geq \frac{1}{3}$. Because $L(f) \leq U(f)$, we have that $L(f) = U(f) = \frac{1}{3}$.

    Since $L(f) = U(f)$, this function is Riemann-integrable. Therefore, $$\int_0^1 x^2 = \int_0^1 x^2dx = \frac{1}{3}$$
\end{example}

\begin{theorem}
    Let $f$ be a bounded function on $[a, b]$. Then, $f$ is Riemann-integrable if and only if given an $\epsilon > 0$, $\exists$ a partition of $[a, b]$ such that $$U(f, \mathcal{P}) - L(f, \mathcal{P}) < \epsilon$$
\end{theorem}
\begin{proof}
    If $f$ is Riemann-integrable, since $\epsilon > 0$, $\exists$ a partition $\mathcal{P}_1$ such that $$L(f, \mathcal{P}_1) > L(f) - \frac{\epsilon}{2}$$
    Similarly, $\exists \mathcal{P}_2$ such that $$U(f, \mathcal{P}_2) < U(f) + \frac{\epsilon}{2}$$
    Let $\mathcal{P} = \mathcal{P}_1 \cup \mathcal{P}_2$. Then,
    \begin{align*}
        U(f, \mathcal{P}) - L(f, \mathcal{P}) &\leq U(f, \mathcal{P}_2) - L(f, \mathcal{P}_1) \\
        &< \left(U(f) + \frac{\epsilon}{2}\right) - \left((L(f) - \frac{\epsilon}{2})\right) \\
        &= U(f) - L(f) + \epsilon \\
        &= \epsilon
    \end{align*}
    Therefore, $f$ is Riemann-integrable.

    Conversely, given $\epsilon > 0$, suppose $\exists \mathcal{P}$ such that $U(f, \mathcal{P}) < L(f, \mathcal{P}) + \epsilon$. Then, $$U(f, \mathcal{P}) \leq U(f, \mathcal{P}) < L(f, \mathcal{P}) + \epsilon \leq L(f) + \epsilon$$
    Therefore, $U(f) \leq L(f)$. But then $L(f) = U(f)$, so $f$ is Riemann-integrable.
\end{proof}
\begin{remark}
    Generally, we just need to find some partition $\mathcal{P}$ such that $U(f, \mathcal{P})$ and $L(f, \mathcal{P})$ are within $\epsilon$ of each other.
\end{remark}
