\chapter{September 15}

\section{Cauchy Sequences}
\begin{definition}{Cauchy sequence}{}
    A sequence of real numbers $S_n$ is called a Cauchy sequence if and only if for each $\epsilon > 0$, $\exists N$ such that $m, n > N \implies |S_m - S_n| < \epsilon$.
\end{definition}
\begin{note}
    This means the elements of the sequence get closer to each other as $N$ increases.
\end{note}

\begin{theorem}{}{}
    Every convergent sequence is Cauchy.
\end{theorem}
\begin{proof}
    Let $S_n$ be a convergent sequence. Then $\exists N$ such that $n > N \implies |S_n - s| < \frac{\epsilon}{2}$ for some $s \in \R$. Then, for $n, m > N$, we have
    \begin{align*}
        |S_n - S_m| &= |S_n - s + s - S_m| \\
        &\leq |S_n - s| + |s - S_m| \\
        &< \frac{\epsilon}{2} + \frac{\epsilon}{2} \\
        &= \epsilon
    \end{align*}
    Thus, $S_n$ is Cauchy.
\end{proof}

\begin{theorem}{}{}
    A sequence of real numbers is Cauchy if and only if it is convergent.
\end{theorem}
\begin{note}
    We cannot prove this yet.
\end{note}