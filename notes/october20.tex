\section{October 20}

\subsection{Mean Value Theorem}
\begin{theorem}[Rolle's theorem]
    Let $f$ be continuous on $[a,b]$ and differentiable on $(a,b)$, and let $f(a) = f(b)$. Then $\exists c \in (a,b)$ such that $f'(c) = 0$.
\end{theorem}
\begin{proof}
    Since $f$ is continuous and $[a, b]$ is compact, $\exists x_1, x_2 \in [a, b]$ such that $f(x_1) \leq f(x) \leq f(x_2) \forall x \in [a, b]$. If $x_1$ and $x_2$ are the endpoints of the interval, then $f$ is a compact function, thus $f'(c) = 0 \forall c \in (a, b)$. Otherwise, $f$ contains a max at $x_2 \therefore f'(x_2) = 0$. Thus $\exists c \in (a, b)$ such that $f'(c) = 0$.
\end{proof}

\begin{theorem}[Mean value theorem]
    Let $f$ be continuous on $[a,b]$ and differentiable on $(a,b)$. Then $\exists c \in (a,b)$ such that $f'(c) = \frac{f(b) - f(a)}{b - a}$.
\end{theorem}
\begin{proof}
    Let $g(x)$ be defined as $g(x) = \frac{f(b) - f(a)}{b - a} (x - a) + f(a)$. Let $h(x)$ be the distance from the graph of $f \circ g$. That is, $h = f - g$. Then, $h$ is continuous on $[a, b]$ and differentiable on $(a, b)$. Furthermore, $h(a) = h(b) = 0$.
    
    By Rolle's Theorem, $\exists c \in (a, b)$ such that $h'(c) = 0$. Thus, $$0 = h'(c) = f'(c) - g'(c) = f'(c) - \frac{f(b) - f(a)}{b - a}$$ Therefore, $f'(c) = \frac{f(b) - f(a)}{b - a}$.
\end{proof}

\begin{theorem}
    Let $f$ be continuous on $[a,b]$ and differentiable on $(a,b)$. Then if $f'(x) = 0 \forall x \in (a, b)$, then $f$ is constant on $[a, b]$.
\end{theorem}
\begin{proof}
    Suppose $f$ is not constant. Then, $\exists x_1, x_2$ such that $a \leq x_1 < x_2 \leq b$ and $f(x_1) \neq f(x_2)$. By the Mean Value Theorem, $\exists c \in (x_1, x_2)$ such that $$f'(c) = \frac{f(x_2) - f(x_1)}{x_2 - x_1} \neq 0$$ But, this is a contradiction. Therefore, $f$ is constant on $[a, b]$.
\end{proof}

\begin{theorem}
    Let $f$ be differentiable on an interval $I$. If $f'(x) > 0 \forall x \in I$, then $f$ is strictly increasing on $I$.
\end{theorem}
\begin{proof}
    Suppose $f'(x) > 0 \forall x \in I$ and $x_1, x_2 \in I$ such that $x_1 < x_2$. Mean Value Theorem implies that $\exists c \in (x_1, x_2)$ such that $f'(c) = \frac{f(x_2) - f(x_1)}{x_2 - x_1}$. Which is to say that $$f(x_2) - f(x_1) = f'(c) (x_2 - x_1)$$ Thus, $f(x_2) - f(x_1)$ is positive since $f'(c)$ and $(x_2 - x_1)$ are both positive. Therefore, $f$ is increasing.
\end{proof}

\subsection{Intermediate Value Theorem}
\begin{theorem}[Intermediate value theorem]
    Let $f$ be continuous on $[a,b]$ and suppose $f(a) < 0 < f(b)$. Then $\exists c \in (a,b)$ such that $f(c) = 0$.
\end{theorem}
\begin{proof}
    Let $c$ be the largest value for which $f(x) \leq 0$. Let $S = \{x \in [a, b] \mid f(x) \leq 0\}$. Since $a \in S, S$, is nonempty. Thus, $\sup S = c$ exists.

    We claim that $f(c) = 0$. Suppose $f(c) < 0$, then $\exists$ a neighborhood $U$ of $c$ such that $f(x) < 0 \forall x \in U \cap [a, b]$. Now, $c \neq b$ since $f(a) < 0 < f(b)$. Thus, $U$ contains a point $p$ such that $c < p < b$ where $f(p) < 0$. But, this is a contradiction since $p \in S$ and $p > c$. Therefore, $f(c) \not< 0$.

    Similarly, suppose $f(c) > 0$. We can follow this proof in the other direction to show that $f(c) = 0$.
\end{proof}