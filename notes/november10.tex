\section{November 10}

\subsection{Fundamental Theorem of Calculus}
\begin{definition}[Fundamental theorem of calculus]
    $f(x): D \to \mathbb{R}$ is uniformly continuous if and only if given $\varepsilon > 0$, $\exists \delta$ such that $|x - y| < \delta \implies |f(x) - f(y)| < \varepsilon$.
\end{definition}
\begin{remark}
    We are closer to proving this!
\end{remark}

\begin{theorem}
    A continuous function on a closed interval $[a, b]$ is uniformly continuous.
\end{theorem}
\begin{remark}
    To be proved.
\end{remark}

\begin{theorem}
    Let $f$ be continuous on $[a, b]$. Then, $f$ is Riemann integrable on $[a, b]$.
\end{theorem}
\begin{proof}
    Since $f$ is continuous on $[a, b]$, $\exists \delta > 0$ such that when $|x - y| < \delta$, $|f(x) - f(y) < \frac{\varepsilon}{b - a} \forall \varepsilon > 0$. Let $\mathcal{P}$ be a partition of $[a, b]$ such that $\Delta x_i < \delta \forall i$. On each subinterval $[x_i, x_{i + 1}]$, $f$ will obtain a maximum and minimum value at $s_i$ and $t_i$ respectively. Furthermore, $|s_i - t_i| < \delta$, so $$0 \leq M_i - m_i = f(t_i) - f(s_i) < \frac{\varepsilon}{b - a} \forall i$$ Then, $$U(f, \mathcal{P}) - L(f, \mathcal{P}) = \sum_{i = 1}^n (M_i - m_i) \Delta x_i < \sum_{i = 1}^n \frac{\varepsilon}{b - a} \Delta x_i = \frac{\varepsilon}{b - a} (b - a) = \varepsilon$$
\end{proof}

\begin{theorem}
    If $f \in R[a, c]$ and $f \in R[c, b]$, then $f \in R[a, b]$ and $$\int_a^b f = \int_a^c f + \int_c^b f$$
\end{theorem}
\begin{proof}
    Given $\varepsilon > 0$, $\exists$ a partition $\mathcal{P}_1$ of $[a, c]$ and $\mathcal{P}_2$ of $[c, b]$ such that $U(f, \mathcal{P}_1) - L(f, \mathcal{P}_1) < \frac{\varepsilon}{2}$ and $U(f, \mathcal{P}_2) - L(f, \mathcal{P}_2) < \frac{\varepsilon}{2}$. Then, define $\mathcal{P} = \mathcal{P}_1 \cup \mathcal{P}_2$. Then, $\mathcal{P}$ is a partition of $[a, b]$ and
    \begin{align*}
        U(f, \mathcal{P}) - L(f, \mathcal{P}) &= U(f, \mathcal{P}_1) + U(f, \mathcal{P}_2) - L(f, \mathcal{P}_1) - L(f, \mathcal{P}_2) \\
        &= U(f, \mathcal{P}_1) - L(f, \mathcal{P}_1) + U(f, \mathcal{P}_2) - L(f, \mathcal{P}_2) \\
        &< \frac{\varepsilon}{2} + \frac{\varepsilon}{2} \\
        &= \varepsilon
    \end{align*}
    So, $f \in R[a, b]$. Furthermore,
    \begin{align*}
        \int_a^b f \leq U(f, \mathcal{P}) &= U(f, \mathcal{P}_1) + U(f, \mathcal{P}_2) \\
        &< L(f, \mathcal{P}_1) + L(f, \mathcal{P}_2) + \varepsilon \\
        &\leq \int_a^c f + \int_c^b f + \varepsilon
    \end{align*}
    Similarly,
    \begin{align*}
        \int_a^b f \geq L(f, \mathcal{P}) &= L(f, \mathcal{P}_1) + L(f, \mathcal{P}_2) \\
        &> U(f, \mathcal{P}_1) + U(f, \mathcal{P}_2) - \varepsilon \\
        &\geq \int_a^c f + \int_c^b f - \varepsilon
    \end{align*}
    Therefore, $\int_a^b f = \int_a^c f + \int_c^b f$.
\end{proof}

\begin{theorem}
    If $f$ is Riemann integrable on $[a, b]$ and $g$ is continuous on $[c, d]$ when $f([a, b]) \subseteq [c, d]$, then $g \circ f$ is Riemann integrable on $[a, b]$.
\end{theorem}
\begin{remark}
    To be proved.
\end{remark}

\begin{theorem}
    Let $f$ be Riemann integrable on a closed interval $[a, b]$. Then, $|f|$ is Riemann integrable on $[a, b]$ and $$\left|\int_a^b f\right| \leq \int_a^b \left|f\right|$$
\end{theorem}
\begin{proof}
    $|x|$ is continuous so we can apply the previous theorem. Then,
    \begin{align*}
        -|f(x)| &\leq f(x) \leq |f(x)| \\
        -\int_a^b |f| &\leq \int_a^b f \leq \int_a^b |f| \text{ because } \frac{M_i}{m_i}
    \end{align*}
    Thus, $\left|\int_a^b f\right| \leq \int_a^b \left|f\right|$.
\end{proof}