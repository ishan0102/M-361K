\chapter{September 29}

\section{Sums of Limits}
\begin{theorem}{}{}
    Let $lim_{x \to c} f(x) = L$, $lim_{x \to c} g(x) = M$. Then, $lim_{x \to c} (f + g)(x) = L + M$.
\end{theorem}
\begin{proof}[Proof (Definition 9.1)]
    Given $\varepsilon > 0$, let $\delta_1 > 0$ be such that $0 < |x - c| < \delta_1 \implies |f(x) - L| < \frac{\varepsilon}{2}$. Let $\delta_2 > 0$ be such that $0 < |x - c| < \delta_2 \implies |g(x) - M| < \frac{\varepsilon}{2}$.

    Let $\delta = min\{\delta_1, \delta_2\}$. Then, for $0 < |x - c| < \delta$, we have $$|f(x) + g(x) - (L + M)| = |(f(x) - L) + (g(x) - M)| \leq |f(x) - L| + |g(x) - M| < \frac{\varepsilon}{2} + \frac{\varepsilon}{2} = \varepsilon.$$
\end{proof}
\begin{proof}[Proof (Theorem 9.3)]
    Let $lim_{x \to c} f(x) = L$, $\lim_{x \to c} g(x) = M$, and $S_n$ be a sequence of real numbers such that $S_n \to c$. Then, $$\lim_{n \to \infty} (f + g)(S_n) = \lim_{n \to \infty} f(S_n) + g(S_n) = \lim_{n \to \infty} f(S_n) + \lim_{n \to \infty} g(S_n) = L + M$$ Thus, $\lim_{x \to c} (f + g)(x) = L + M$.
\end{proof}
\begin{note}
    This is true for $-$, $\times$, and $\div$ as well.
\end{note}

\begin{definition}{Sequential criterion for functional limits}{}
    $\lim_{x \to c} f(x) = L$ if and only if whenever $S_n \to c$, $\lim_{n \to \infty} f(S_n) = L$.
\end{definition}

\begin{theorem}{}{}
    Let $k \in \R$. If $\lim_{x \to c} f(x) = L$, then $\lim_{x \to c} kf(x) = kL$.
\end{theorem}
\begin{proof}
    Let $\lim_{x \to c} f(x) = L$, $k \in \R$, and $S_n$ be a sequence of real numbers such that $S_n \to c$. Then, $$\lim_{n \to \infty} kf(S_n) = k\lim_{n \to \infty} f(S_n) = kL$$ Thus, $\lim_{x \to c} kf(x) = kL$.
\end{proof}

\section{Continuity of Functions}
\begin{definition}{Continuous function}{}
    A function $f$ is continuous at $x = c$ if and only if $\lim_{x \to c} f(x) = f(c)$. Let $s$ be an accumulation point of the domain $f:D \to \R$. Then, $f$ is continuous at $s$ if and only if for each $\varepsilon > 0$, $\exists \delta > 0$ such that whenever $0 < |x - s| < \delta$, $|f(x) - f(s)| < \varepsilon$.
\end{definition}
\begin{note}
    Let $f(x) = x\sin(\frac{1}{x})$ where $x \neq 0$, $f(0) = 0$. If we want to show that this function is continuous, we need to find some $\delta > 0$ such that $|x| < \delta \implies |f(x) - f(0)| < \varepsilon$. Let $\delta = \varepsilon$, then when $|x| < \delta$, $|f(x) - f(0)| = |x\sin(\frac{1}{x}) - 0| = |x\sin(\frac{1}{x})| \leq |x| < \varepsilon$.
\end{note}

\begin{theorem}{}{}
    If $f$ and $g$ are continuous at $x = c$, then $f + g$ is also continuous at $x = c$.
\end{theorem}
\begin{proof}
    Let $f$ and $g$ be continuous at $c$ and $S_n$ be a sequence of real numbers such that $S_n \to c$. Then, $$\lim_{n \to \infty} (f + g)(S_n) = \lim_{n \to \infty} f(S_n) + \lim_{n \to \infty} g(S_n) = f(c) + g(c)$$ Thus, $\lim_{x \to c} (f + g)(x) = (f + g)(c)$.
\end{proof}

\begin{theorem}{}{}
    Let $f:D \to E$ be continuous at $x = c$ and let $g:E \to R$ be continuous at $x = f(c)$. Then, the composition $g \circ f$ is continuous at $x = c$.
\end{theorem}
\begin{proof}
    This is left as an exercise for the reader.
\end{proof}