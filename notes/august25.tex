\section{August 25}
\subsection{Algebraic Axioms} $\forall a, b, c \in \R$
\begin{itemize}
    \item (A1) $a + b = b + a$.
    \item (A2) $(a + b) + c = a + (b + c)$.
    \item (A3) $\exists$ an element $o \in \R$ such that $a + o = o + a = a$.
    \item (A4) For each element $a \in \R$, $\exists$ an element $(-a) \in \R$ such that $a + (-a) = 0$.
    \item (M1) $ab = ba$.
    \item (M2) $(ab)c = a(bc)$.
    \item (M3) $\exists$ an element $1 \in \R$ such that $a * 1 = 1 * a = a$.
    \item (M4) For each element $a \in \R \setminus {0}$, $\exists$ an element $\frac{1}{a} \in \R$ such that $a * \frac{1}{a} = \frac{1}{a} * a = 1$.
    \item (D) $a * (b + c) = a * b + a * c$.
\end{itemize}
\begin{note}
    If $a = b$ and $c = d$, then $a + c = b + d$ and $a * c = b * d$.
\end{note}

$\forall x, y, z \in \R$:
\begin{theorem}{}{}
    If $x + z$ = $y + z$ then $x = y$.
\end{theorem}
\begin{proof}
    \begin{align*}
        x + z &= y + z \;\; (A4) \\
        (x + z) + (-z) &= (y + z) + (-z) \;\; (A2) \\
        x + (z + (-z)) &= y + (z + (-z)) \;\; (A4) \\
        x + 0 &= y + 0 \;\; (A3) \\
        x &= y
    \end{align*}
\end{proof}

\begin{theorem}{}{}
    For any $x \in \R$, $x * 0 = 0$.
\end{theorem}
\begin{proof}
    \begin{align*}
        x * 0 &= x * (0 + 0) \\
        x * 0 &= x * 0 + x * 0 \\
        x * 0 + (-x * 0) &= (x * 0 + x * 0) + (-x * 0) \\
        0 &= x * 0 + (x * 0 + (-x * 0)) \\
        &= x * 0 + 0 \\
        &= x * 0
    \end{align*}
\end{proof}

\begin{theorem}{}{}
    $-1 * x = -x$ i.e. $x + (-1) * x = 0$.
\end{theorem}
\begin{proof}
    \begin{align*}
        x + (-1) * x &= x + x * (-1) \\
        &= x * 1 + x * (-1) \\
        &= x * (1 + (-1)) \\
        &= x * 0 \\
        &= 0
    \end{align*}
\end{proof}

\begin{theorem}{Zero-product property}{}
    $\forall x, y \in \R$, $x * y = 0 \iff x = 0 \lor y = 0$.
\end{theorem}
\begin{proof}
    Let $x, y \in \R$, if $x = 0$ or $y = 0$, then $x * y = 0$. Suppose $x \neq 0$, then we must show $y = 0$. Since $x \neq 0$, $\frac{1}{x}$ exists. Thus, if:
    \begin{align*}
        xy &= 0 \\
        \frac{1}{x} * (xy) &= \frac{1}{x} * 0 \\
        (\frac{1}{x} * (xy)) * y &= 0 \\
        1 * y &= 0 \\
        y &= 0
    \end{align*}
\end{proof}

\subsection{Order Axioms} $\forall x, y \in \R$:
\begin{itemize}
    \item (O1) One of $x < y$, $x > y$ or $x = y$ is true.
    \item (O2) If $x < y$ and $y < z$, then $x < z$.
    \item (O3) If $x < y$ then $x + z < y + z$.
    \item (O4) If $x < y$ and $z > 0$ then $xz < yz$.
\end{itemize}

\begin{theorem}{}{}
    If $x < y$ then $-y < -x$.
\end{theorem}
\begin{proof}
    \begin{align*}
        x &< y \\
        x + (-x + -y) &< y + (-x + -y) \\
        (x + -x) + -y &< (y + -y) + -x \\
        0 + -y &< 0 + -x \\
        -y &< -x
    \end{align*}
\end{proof}

\begin{theorem}{}{}
    If $x < y$ and $z > 0$ then $xz > yz$.
\end{theorem}
\begin{proof}
    If $x < y$ and $z > 0$ then $-z < 0$. Thus, $x(-z) < y(-z)$. But,
    \begin{align*}
        x(-z) &= x(-1 * z) \\
        &= (x * -1) * z \\
        &= (-1 * x) * z \\
        &= -1 (x * z) \\
        &= -x * z
    \end{align*}
    Similarly, $y(-z) = -y * z$. Thus, $-x * z < -y * z$, so $xz > yz$.
\end{proof}

\begin{note}
    $\R$ is an ordered field. $\R$ is complete, while $\Q$ is not complete.
\end{note}