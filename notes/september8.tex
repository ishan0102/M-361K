\section{September 8}

\subsection{Limits of Sequences}
\begin{definition}{Limit of a sequence}{}
    A sequence $a_n$ is said to converge to a real number $s$, if for any $\epsilon > 0$, $\exists$ a real number $k$ such that for all $n \geq k$, the terms $a_n$ satisfy $|a_n - s| < \epsilon$.
\end{definition}

\begin{theorem}{}{}
    $\lim_{n \to \infty} \frac{1}{\sqrt{n}} = 0$.
\end{theorem}
\begin{proof}
    We need to find some $N$ such that $n > N \forall \epsilon > 0$.
    \begin{align*}
        | \frac{1}{\sqrt{n}} - 0 | &< \epsilon \\
        \frac{1}{\sqrt{n}} &< \epsilon \\
        \frac{1}{n} &< \epsilon^2 \\
        n &> \frac{1}{\epsilon^2}
    \end{align*}
    
    Let $\epsilon > 0$ and $N = \frac{1}{\epsilon^2}$. Then, if $n > N$, we have that
    \begin{align*}
        | \frac{1}{\sqrt{n}} - 0 | &= \frac{1}{\sqrt{n}} \\
        &< \frac{1}{\sqrt{\frac{1}{\epsilon^2}}} \\
        &= \epsilon
    \end{align*}

    Thus, $\lim_{n \to \infty} \frac{1}{\sqrt{n}} = 0$.
\end{proof}

\begin{theorem}{}{}
    $\lim_{n \to \infty} 1 + \frac{1}{2^n} = 1$.
\end{theorem}
\begin{proof}
    Let $\epsilon > 0$ and $N = \frac{1}{\epsilon}$. Then, we have
    \begin{align*}
        | 1 + \frac{1}{2^n} - 1 | &< \epsilon \\
        | \frac{1}{2^n} | &= \frac{1}{2^n} < \frac{1}{n} < \frac{1}{\frac{1}{\epsilon}} < \epsilon \\
        n &> \frac{1}{\epsilon}
    \end{align*}

    Thus, $\lim_{n \to \infty} 1 + \frac{1}{2^n} = 1$.
\end{proof}

\begin{theorem}{}{}
    Every convergent sequence is bounded.
\end{theorem}
\begin{proof}
    Let $S_n$ be a convergent sequence with a limit $s$ and $\epsilon = 1$. Then, there exists some $N$ such that $|S_n - s| < 1$ when $n > N$. That is, $|S_n| < |s| + 1$.

    Let $M = \max\{S_1, S_2, \ldots, S_n, |s| + 1\}$. Then, $|S_n| \leq M$, so $S_n$ is bounded.
\end{proof}

\begin{theorem}{}{}
    If a sequence converges, its limit is unique.
\end{theorem}
\begin{proof}
    Suppose a sequence $S_n$ converges to $s$ and $t$. Let $\epsilon > 0$. Then, $\exists N_1$ such that $|S_n - s| < \frac{\epsilon}{2}$. For $n > N_1$, $\exists N_2$ such that $|S_n - t| < \frac{\epsilon}{2}$. For $n > N_2$, let $N = m + \{N_1, N_2\}$. Then, for $n > N$, we have
    \begin{align*}
        |s - t| &= |s + S_n - S_n - t| \\
        &= |s - S_n + S_n - t| \\
        &\leq |s - S_n| + |S_n - t| \\
        &< \frac{\epsilon}{2} + \frac{\epsilon}{2} \\
        |s - t| &= \epsilon
    \end{align*}

    Thus, the limit is unique.
\end{proof}